\documentclass[10pt]{article}
\usepackage[english]{babel}
\usepackage[utf8]{inputenc}
\usepackage[T1]{fontenc}

% % -------------------------------------------------------------------
% % Pacotes matemáticos
% \usepackage{amsmath,amsfonts,amssymb,amsthm,cancel,siunitx,
% calculator,calc,mathtools,empheq,latexsym}
% % -------------------------------------------------------------------
% Pacotes para inserção de figuras e subfiguras
% \usepackage{subfig,epsfig,tikz,float}		            % Packages de figuras.
% -------------------------------------------------------------------
% Pacotes para inserção de tabelas
% \usepackage{booktabs,multicol,multirow,tabularx,array}          % Packages para tabela
% \usepackage{natbib}
% -------------------------------------------------------------------
% Definição de comprimentos

\setlength{\parindent}{0pt}
\setlength{\parskip}{5pt}
\textwidth 13.5cm
\textheight 19.5cm
\columnsep .5cm

% \title{
\title{\renewcommand{\baselinestretch}{1.15}
Stations: a decentralized social media platform
}

\author{%
Tony Fischetti
}



%Início do documento

\begin{document}

\date{}

\maketitle

\vspace{-1.2cm}

\begin{center}
{\footnotesize
tony.fischetti@gmail.com
}
\end{center}

% -------------------------------------------------------------------
% Abstract
\bigskip
\noindent
{\small{\bf Abstract.}

Decentralized applications (dapps) and the public, permissionless blockchains
on which are built, promise a paradigm shift in how we interact with the web
and offers a compelling alternative to centralized media brokers and their
pitfalls.
Up until now, however, the vast majority of dapps relate to financial
applications. To realize Web 3.0's promise, dapps have to expand their reach
into other (non self-referential) domains and, in particular, those that
have made the immoderation of current Web 2.0 model most manifest.

In this paper, we discuss what a decentralized social media platform might
look like, explore different implementation strategies and their respective
tradeoffs, and, finally, describe a novel approach based on "smart contracts"
and single page javascript applications.

}

\medskip
\noindent
{\small{\bf Keywords}{:}
decentralized computing, blockchain, social media
}

\baselineskip=\normalbaselineskip
% -------------------------------------------------------------------

\section{Introduction}\label{sec:1}

\subsection{Web 2.0 and its harms}

\subsection{Challenges presented by decentralization}

\section{Different approaches}\label{sec:2}

\subsection{Previous work}

\subsection{Peer to peer approaches}

\section{A web-based numbers station}

Taking inspiration from the cold war era....


\section{Results}\label{sec:3}
we

\section{Conclusions}\label{sec:4}
boys

\newpage

\end{document}


something akin to a web-based "numbers station"

herein the broadcaster can (but need not) remain anonymous

and nothing short of jamming all shortwave frequencies can prevent
others from listening in.


Blockchains (at least as it concerns this project) can be thought of as
immutable, append-only, highly available, distributed database capable of
arbitrary code execution.

